\documentclass{article}
\usepackage{multicol}
\usepackage{array}
\usepackage{makeidx}
\usepackage[skaknew]{chessfss}
\usepackage{texmate}
\usepackage{xskak}
\usepackage[top=1.5cm, bottom=2cm, left=1.5cm, right=1cm,headheight=15pt]{geometry}
\usepackage{adjmulticol}
\usepackage{ragged2e}

\begin{document}
\chessevent{Vienna}

\chessopening{Vienna}

Date : 1907.??.??

EventDate : ?

Round : ?

Result : 1/2-1/2

\whitename{Rudolf Spielmann}

\blackname{Geza Maroczy}

\ECO{C25}

\whiteelo{?}

\blackelo{?}

PlyCount : 88

\makegametitle
\begin{multicols}{2}
\noindent
\newchessgame[id=main]
\xskakset{style=styleC}
\mainline{1. e4  e5 2. Nc3  Bc5 3. Bc4  d6 4. d3  Nf6 }
\xskakcomment{\small\texttt\justifying{\textcolor{darkgray}{~  test le commentaire qui peut être un peu long et stable malgré tout ! test le commentaire qui peut être un peu long et stable malgré tout ! C'est le test du commentaire qui peut être un peu long et stable malgré tout ! test le commentaire qui peut être un peu long et stable malgré tout ! test le commentaire qui peut être un peu long et stable malgré tout ! test le commentaire qui peut être un peu long et stable malgré tout ! }}} 
\mainline{}
\scalebox{0.90}{\chessboard}\\
\mainline{5. f4  Bg4 6. Nf3  Nc6 7. Na4  Nd7 8. Nxc5  dxc5 }
\scalebox{0.90}{\chessboard}\\
\mainline{9. O-O  exf4 10. Bxf4  Nce5 11. Nxe5  Bxd1 12. Nxf7  Qf6 }
\scalebox{0.90}{\chessboard}\\
\mainline{13. Raxd1  Rf8 14. Bxc7  Qxb2 15. Rf2  Nf6 16. Be5  Qb6 }
\scalebox{0.90}{\chessboard}\\
\mainline{17. Nd6+  Kd7 18. d4  Ng4 19. Rfd2  Qb4 20. c3  Qxc3 }
\scalebox{0.90}{\chessboard}\\
\mainline{21. Rd3  Qb2 22. R3d2  Qb4 23. a3  Qxa3 24. dxc5  Qe3+ }
\scalebox{0.90}{\chessboard}\\
\mainline{25. Kh1  Nxe5 26. Nf5+  Qxd2 27. Rxd2+  Kc7 28. Bd5  Rad8 }
\scalebox{0.90}{\chessboard}\\
\mainline{29. h3  Rd7 30. Nd4  Rf1+ 31. Kh2  a6 32. Rb2  Nc6 }
\scalebox{0.90}{\chessboard}\\
\mainline{33. Ne6+  Kc8 34. Bxc6  bxc6 35. Rb6  Rf6 36. Rxc6+  Kb7 }
\scalebox{0.90}{\chessboard}\\
\mainline{37. Rb6+  Ka7 38. e5  Rh6 39. Rd6  Re7 40. Nd4  Kb7 }
\scalebox{0.90}{\chessboard}\\
\mainline{41. c6+  Kc7 42. Nf5  Rxe5 43. Nxh6  gxh6 44. Rxh6  Re7 } Score : 1/2-1/2
\end{multicols}
\newpage
\chessevent{Goteborg}

\chessopening{11}

Date : 1920.??.??

EventDate : ?

Round : ?

Result : 1-0

\whitename{Rudolf Spielmann}

\blackname{Jorgen Moeller}

\ECO{C33}

\whiteelo{?}

\blackelo{?}

PlyCount : 55

\makegametitle
\begin{multicols}{2}
\noindent
\newchessgame[id=main]
\xskakset{style=styleC}
\mainline{1. e4  e5 2. f4  exf4 3. Qf3  Nc6 4. c3  Nf6 }
\scalebox{0.90}{\chessboard}\\
\mainline{5. d4  d5 6. e5  Ne4 7. Bb5  Qh4+ 8. Kf1  g5 }
\scalebox{0.90}{\chessboard}\\
\mainline{9. Nd2  Bg4 10. Nxe4  Bxf3 11. Nxf3  Qh6 12. Nf6+  Kd8 }
\scalebox{0.90}{\chessboard}\\
\mainline{13. h4  Be7 14. Nxg5  Qg6 15. Nxd5  Bxg5 16. hxg5  Qc2 }
\scalebox{0.90}{\chessboard}\\
\mainline{17. Be2  Ne7 18. Nxf4  c5 19. Rh3  cxd4 20. Rd3  Kd7 }
\scalebox{0.90}{\chessboard}\\
\mainline{21. Bd1  Qxd3+ 22. Nxd3  dxc3 23. bxc3  Rad8 24. Be2  Nf5 }
\scalebox{0.90}{\chessboard}\\
\mainline{25. Bf4  Kc7 26. Rb1  b6 27. e6+  Kc8 28. Ne5 } Score :  1-0 
\end{multicols}
\newpage
\chessevent{San Sebastian}

\chessopening{10}

Date : 1912.??.??

EventDate : ?

Round : ?

Result : 0-1

\whitename{Akiba Rubinstein}

\blackname{Rudolf Spielmann}

\ECO{A84}

\whiteelo{?}

\blackelo{?}

PlyCount : 84

\makegametitle
\begin{multicols}{2}
\noindent
\newchessgame[id=main]
\xskakset{style=styleC}
\mainline{1. d4 }
\xskakcomment{\small\texttt\justifying{\textcolor{darkgray}{~  Notes by Jacques Mieses and Dr. Savielly Tartakower.}}} 
\mainline{1...  e6 }
\xskakcomment{\small\texttt\justifying{\textcolor{darkgray}{~  ESSAI e6}}} 
\mainline{2. c4  f5 3. Nc3  Bb4 4. Bd2  Nf6 }
\scalebox{0.90}{\chessboard}\\
\mainline{5. g3  O-O 6. Bg2  d6 7. a3  Bxc3 8. Bxc3  Nbd7 }
\scalebox{0.90}{\chessboard}\\
\mainline{9. Qc2  c5 10. dxc5  Nxc5 11. Nf3  Nce4 12. O-O  Bd7 }
\scalebox{0.90}{\chessboard}\\
\mainline{13. Rfd1 }
\xskakcomment{\small\texttt\justifying{\textcolor{darkgray}{~  ?
Mieses: White should play 14.Rad1. As the game shows, the
f2-pawn must be protected.}}} 
\mainline{13...  Rc8 14. Bxf6  Qxf6 15. Qb3  Rc7 16. Ne1 }
\xskakcomment{\small\texttt\justifying{\textcolor{darkgray}{~  ?! Mieses: Better was 16.Nd4. *** Tartakower: If 16.Nd4,
then, as in the text, 16...Nc5 17.Qb4 f4, etc. If 16.e3 e5.}}} 
\mainline{16...  Nc5 }
\scalebox{0.90}{\chessboard}\\
\mainline{17. Qb4  f4 }
\xskakcomment{\small\texttt\justifying{\textcolor{darkgray}{~  !}}} 
\mainline{18. Nd3 }
\xskakcomment{\small\texttt\justifying{\textcolor{darkgray}{~  Tartakower: If 18.Rxd6 fxg3 19.fxg3
Qf2+ 20.Kh1 Qxe2 21.h3 (21.Rd2 Rf1+ followed by mate) 21...Ne4
and wins.}}} 
\mainline{18...  fxg3 19. fxg3  Nxd3 20. Rxd3  Qf2+ }
\scalebox{0.90}{\chessboard}\\
\mainline{21. Kh1  Bc6 }
\xskakcomment{\small\texttt\justifying{\textcolor{darkgray}{~  !}}} 
\mainline{22. e4  Rcf7 23. Re1 }
\xskakcomment{\small\texttt\justifying{\textcolor{darkgray}{~  Tartakower: If 23.Rxd6 Qe2 24.Qe1 (still parrying
the triple threat ...Rf1+ or ...Bxe4 or ...Rf2) 24...Qxb2
25.Rxe6 Rf2, etc.}}} 
\mainline{23...  a5 }
\xskakcomment{\small\texttt\justifying{\textcolor{darkgray}{~  !}}} 
\mainline{24. Qc3  Qc5 }
\scalebox{0.90}{\chessboard}\\
\mainline{25. b4  Bxe4 }
\xskakcomment{\small\texttt\justifying{\textcolor{darkgray}{~  ! Mieses:
Brilliant and correct!}}} 
\mainline{26. Rxe4 }
\xskakcomment{\small\texttt\justifying{\textcolor{darkgray}{~  Mieses: The queen cannot be
captured because of mate in two moves. After 26.Bxe4 the
following winning continuation for Black is shown: 26...Rf1+
27.Rxf1 Rxf1+ 28.Kg2 Rg1+! 29.Kf3 Qh5+ 30.Ke3 Qxh2, etc. But,
as pointed out in "Deutsche Schachblaetter", playing 26.Rf3!
White has the opportunity for strong resistance with big
drawing chances; for example: 26...Rxf3 27.Qxf3! Rxf3 28.bxc5,
or 26....Qc6 27.b5! Rxf3 28.Qxf3 Bxf3 29.bxc6 Bxc6 30.Bxc6
bxc6 31.Rxe6, etc.}}} 
\mainline{26...  Rf1+ 27. Bxf1  Rxf1+ 28. Kg2  Qf2+ }
\scalebox{0.90}{\chessboard}\\
\mainline{29. Kh3  Rh1 30. Rf3 }
\xskakcomment{\small\texttt\justifying{\textcolor{darkgray}{~  ! Tartakower: If 30.Rf4 Qxh2+ 31.Kg4 Qh5+ mate, and if
30.Rxe6 Qxh2+ 31.Kg4 Qh3+, followed by 32...Qxe6.}}} 
\mainline{30...  Qxh2+ 31. Kg4  Qh5+ 32. Kf4  Qh6+ }
\scalebox{0.90}{\chessboard}\\
\mainline{33. Kg4  g5 }
\xskakcomment{\small\texttt\justifying{\textcolor{darkgray}{~  ! Tartakower: A splendid
point, establishing a mating net at one stroke.}}} 
\mainline{34. Rxe6 }
\xskakcomment{\small\texttt\justifying{\textcolor{darkgray}{~  Tartakower: If 34.Rf8+ Kxf8 35.Kf3 axb4 36.axb4 Kf7 with
37...Qf6 to follow.}}} 
\mainline{34...  Qxe6+ 35. Rf5  h6 }
\xskakcomment{\small\texttt\justifying{\textcolor{darkgray}{~  Mieses: Enough to win,
but here was also a quicker solution: 35...Qe4+ 36.Kxg5 h6+
37.Kf6 (37.Kg6 Qe8+) 37...Re1! 38.Kg6 Qg4+.}}} 
\mainline{36. Qd3  Kg7 }
\scalebox{0.90}{\chessboard}\\
\mainline{37. Kf3  Rf1+ }
\xskakcomment{\small\texttt\justifying{\textcolor{darkgray}{~  !}}} 
\mainline{38. Qxf1  Qxf5+ 39. Kg2  Qxf1+ 40. Kxf1  axb4 }
\scalebox{0.90}{\chessboard}\\
\mainline{41. axb4  Kf6 42. Kf2  h5 } Score : 0-1
\end{multicols}
\newpage\end{document}